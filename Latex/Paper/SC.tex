% AER-Article.tex for AEA last revised 22 June 2011
\documentclass[AER]{AEA}

% The mathtime package uses a Times font instead of Computer Modern.
% Uncomment the line below if you wish to use the mathtime package:
%\usepackage[cmbold]{mathtime}
% Note that miktex, by default, configures the mathtime package to use commercial fonts
% which you may not have. If you would like to use mathtime but you are seeing error
% messages about missing fonts (mtex.pfb, mtsy.pfb, or rmtmi.pfb) then please see
% the technical support document at http://www.aeaweb.org/templates/technical_support.pdf
% for instructions on fixing this problem.

% Note: you may use either harvard or natbib (but not both) to provide a wider
% variety of citation commands than latex supports natively. See below.

% Uncomment the next line to use the natbib package with bibtex 
%\usepackage{natbib}

% Uncomment the next line to use the harvard package with bibtex
%\usepackage[abbr]{harvard}

% This command determines the leading (vertical space between lines) in draft mode
% with 1.5 corresponding to "double" spacing.
\draftSpacing{1.5}

\begin{document}
	
	\title{Assessing The Limits of Synthetic Controls: \\ Estimating Causal Effects in Complex Data Structures}
	\shortTitle{}
	\author{Lennart Bolwin and Justus Töns\thanks{Thank You Jörg!}}
	\date{\today}
	\pubMonth{Month}
	\pubYear{Year}
	\pubVolume{Vol}
	\pubIssue{Issue}
	\JEL{}
	\Keywords{}
	
	\begin{abstract}
		Potential framework: We argue that applications of Synthetic Controls are faced with a self-selection problem. That is, the method is primarily applied to non-complex data structures that are straightforward to forecast (e.g. development of GDP). Using Monte Carlo studies, we show that the high interpretability of Synthetic Controls comes at the costs of poor in-sample predictions when the data structure exhibits a high degree of complexity like in stock market time series. To address this issue, we introduce the intricacy-statistic that informs the applied reseacher whether or not the data at hand exceeds the level of complexity that Synthetic Controls can handle. If the case, more flexible methodologies like the autoregressive distributed lag model can be employed to estimate the counterfactual. To do so, we introduce the R-package complex\_synths that provides ready-to-use functions to compute the intricacy-statistic and, based on the magnitude of the statistics, the functionalities to estimate either the Synthetic Control or the ARDL model. To probe the performance of our methodology outside the quasi-experimental setting, we apply it to a highly complex data structure: The inclusion of a stock in an index. Specifically, we find that the inclusion of the German multi-national eCommerce company Zalando in the German stock index (DAX) caused an excess capitalization of XXX milion euro. 
	\end{abstract}
	
	
	\maketitle
	
	\section{Introduction}
	
	some text
	
	\section{Research Design}
	
	some text
	
	\section{Monte Carlo Study}
	
	some text
	
	\section{Empirical Application}
	
	some text
	
	\section{Concluding Remarks}
	
	References here (manual or bibTeX). If you are using bibTeX, add your bib file 
	name in place of BibFile in the bibliography command.
	% Remove or comment out the next two lines if you are not using bibtex.
	\bibliographystyle{aea}
	\bibliography{BibFile}
	
	% The appendix command is issued once, prior to all appendices, if any.
	\appendix
	
	\section{Mathematical Appendix}
	
\end{document}

