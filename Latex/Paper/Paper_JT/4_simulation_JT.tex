\section{Simulation}
%\begin{itemize}
%	\item Simulation study is supposed to guide us to the full VARSC-model
%	\item Ideas in OneNote
%\end{itemize}

\subsection{Static Data Generating Processes}

Simulation-Procedure


\subsection{Weakly Dynamic Data Generating Processes}
\subsection{Dynamic Data Generating Processes}
\subsubsection{Set up}
In order to rigorously evaluate the performance of both, the here proposed and existing SC estimators, a key milestone would be to test the estimators in simulated datasets which mimic the real world. As many of the previous studies have focused on the economic development following a treatment, it stands to reason that real world changes in GDP could serve as the basis for a close-to-reality inspired DGP.
To ensure that the DGP is based on a relatively uniform reference group which inhibits significant amounts of commonalities and correlation, the data basis for this simulation consists of all countries from the G20 as well as the European Union (EU). \footnote{The GDP data is sourced from the World Bank's World Development Indicators, which is directly accessible via the WDI-Package in R using the ticker 'NY.GDP.PCAP.KD.ZG' (GDP Per Capita Growth Rate).} 
\textcolor{magenta}{Achtung: R Paket richtig zitieren}\\
The dataset is subjected to two additional filters: firstly, only countries with at least 40 years worth of GDP data remain in the dataset, and secondly, the time series selected must be stationary. The latter is tested using the Augmented Dickey Fuller Test on a 10\% significance level. The remaining 22 countries are the base from which the close-to-reality datasets are simulated using a VAR model.\\
Two different approaches, which will be referred to in the following as the \textit{micro approach} and \textit{macro approach}, are being considered. To evaluate the proposed models in terms of an increasing number of donors as in the static case, the results naturally vary with the underlying base, the data is simulated from. 
It depends critically on whether the simulation is performed with the equal number of donors as the models are estimated with (micro approach), or whether the simulation is performed with a larger number of donors such that the model estimation is carried out with subsets of the whole simulation (macro approach).
The micro approach is a more controlled and stylized design, in which the model estimation is performed on the full simulated dataset. Simulation and estimation are therefore performed using the same amount of $J$ donors in each iteration, i.e. is therefore internally consistent. In this case, adding more donors means adding complexity to the simulation as well as the estimation of the model. The macro approach is a more realistic approach in the sense, that by increasing the number of donors, new information is added to the model estimation. Hence, in the macro approach the dataset is always simulated with $J_{max}$ donors, while the model estimation is performed using $J$ donors, such that for all estimation iterations $J_{max} > J $.  This approach reflects the effects of adding new information in terms of donors to the model estimation while the simulated dataset remains structurally the same.\\
For simplicity, the lag order of the simulation VAR process is set to $p=2$ and to ensure consistency with the static DGP the same $T_{pre}$ and $T_{post}$ values are used in simulation. However due to dimensionality constraints of the VAR model, the number of donors is reduced to $ J \in \left\lbrace 2,4,6,8\right\rbrace$ and $J_{max} = 14$. Therefore, in each iteration  $J+1$ (micro), or $J_{max}+1$ (macro) countries are drawn at random from the 22-country base. One of the countries is randomly assigned to be the treatment unit while the remaining ones function as (potential) donors.

\subsubsection{Employed Models}
For the Dynamic \ac{DGP}, we employ the models of the static case as reference, as well as the following three models reflecting the time series structure of the data. 
\\
VAR => reference model
\\
Unidyn:
\\
Multidyn/ADL Model: Hier Md1 und MD2 nur kurz nennen, Multidyn 3 dann aber als Essenz mit den entsprechenden Vorteilen den anderen gegenüber. Dann in den Results wieder kurz darauf beziehen (Ergebnisse der anderen im Appendix?)
\\
refer to the metrics in the static case => keep short here \\


\subsubsection{Results}
Short desciption of the model specific parameters used \\
divide by the two DGPs and also compare the development when J increases \\
Full Table in Appendix \\
Start with general findings then details regarding the models \\



\textit{Micro View} \\


\textit{Macro View} \\
